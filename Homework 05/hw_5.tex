\documentclass[11pt]{article}

\usepackage[activate={true,nocompatibility},final,tracking=true,kerning=true,spacing=true,factor=1100,stretch=10,shrink=10]{microtype}
\usepackage{multirow}
\usepackage{mathtools}
\usepackage{graphicx}
\usepackage{tabularx}
\usepackage{enumerate}
\usepackage{amsmath, amssymb, graphics, setspace}
\usepackage[noline,noend,ruled,linesnumbered]{algorithm2e}
\usepackage{algpseudocode}
\usepackage{enumitem}
\usepackage{amssymb}
\usepackage{amsmath}
\usepackage{latexsym}
\usepackage{graphics}
\usepackage{multirow}
\usepackage{verbatim}
\usepackage[T1]{fontenc}
\usepackage{subfigure}
\usepackage{tikz}
\usetikzlibrary{arrows}
\usepackage{url}
\usepackage[noline,noend,ruled,linesnumbered]{algorithm2e}
\usepackage{algpseudocode}

\DeclareMathOperator{\lcm}{lcm}

\newcommand{\mathsym}[1]{{}}
\newcommand{\unicode}[1]{{}}

\title{Assignment 5 \\ Minimum Spanning Trees, NP-Complete Problems, and Approximation Algorithms}

\author{Paul Jones and Matthew Klein \\
		Professor Kostas Bekris\\
		Design and Analysis of Computer Algorithms (01.198.344)}

\date{\today}

\usepackage{setspace}
\singlespacing
 \usepackage{url} 
\usepackage[letterpaper]{geometry}
\geometry{top=1in, bottom=1in, left=.5in, right=.5in}

\usepackage{fancyhdr}
\pagestyle{fancy}
\lhead{Paul Jones and Matthew Klein \\ Professor Kostas Bekris}
\chead{}
\rhead{Design and Analysis of Computer Algorithms \\ Rutgers University}
\lfoot{}
\cfoot{\thepage}
\rfoot{}

\begin{document}

\maketitle
\clearpage

\subsection*{Problem 1} You are a programmer in an
electronic commerce company, where you are asked to design a new
recommendation system. The idea is to divide all users into $k$ groups
based on their purchase history. After the grouping, your system can
provide recommendations to users based on the purchases of other users
in the same group.\\

\noindent To perform the grouping, we define a function $s(u_i, u_j)$
to reflect the similarity of two users $u_i$ and $u_j$. The similarity
function is defined as: \vspace{-0.05in}

$$
s(u_i, u_j) = \begin{cases}
  \parbox{2in}{1 + the number of common purchases $u_i$ and $u_j$ have made} & 
  (i \neq j) \\
    \\
  0 & (i = j)
\end{cases}
$$

\noindent To optimize the grouping, we need to minimize the similarity
of the most similar pair of users that have been assigned to different
groups.\\

\begin{enumerate}[label=\textbf{(\alph*)}]
\item \textbf{Provide an efficient algorithm that groups all users
into $k$ groups and satisfies the optimality requirement mentioned
above (i.e., minimizes the similarity of users in different groups).}

\item \textbf{What data structures do you assume for the
implementation of your algorithm?  What is the running time of your
approach given the data structures you employ?}

\item \textbf{Prove the correctness of your algorithm.}
\end{enumerate}

\subsection*{Problem 2 (20 points):} In wireless sensor networks, it
is an important task to periodically collect data from an area of
interest for time-sensitive applications. The sensed data must be
gathered and transmitted from the sensors to a base station through
the most efficient way in terms of energy and speed.\\

\noindent The sensor nodes are deployed randomly in the target
field. Then, the network establishment begins with the formation of
communication links between the sensor nodes. Consider that $E(a,b)$
is the energy needed to send message from node $a$ to node $b$, which
typically depends on the distance between the two nodes.\\

\noindent An idea in the sensor network literature is to use a minimum
spanning tree (MST) as the underlying communication structure in order
to shorten the total transmission distance, while guaranteeing that
there is a way for a message to reach every node in the network. This
means that we need to construct a MST of the nodes in the network.\\

\noindent It is possible, however, that a node in the resulting MST
will be connected with a lot of neighboring nodes, e.g., consider the
case the MST looks like a star-like structure where a central node is
connected to all other nodes. Such nodes with many edges may need to
fuse more data collected from its neighbors than other nodes and to
consume more energy.  This may cause nodes with a lot of neighbors to
die earlier because they exhaust their energy.  In order to avoid this
situation, we limit the number of connections for each node to be at
most $d$ in the communication tree, for a given positive integer
$d$.\\

\begin{enumerate}[label=\textbf{\Alph*.}]
\item \textbf{For $d = 2$, describe an algorithm for finding such
a communication tree and show that this problem is NP-complete
problem.  (Hint: consider the relation with finding a RUDRATA-PATH on
a graph.)}

\item \textbf{Describe a polynomial time algorithm (in detail)
that solves this problem, if you have an algorithm that solves TSP in
a polynomial time. }
\end{enumerate}

\subsection*{Problem 3 (30 points):} Lord Tywin Lannister is
organizing the feast for the upcoming wedding of his grandson, King
Tommen Baratheon, with Lady Margaery Tyrell. The Queen of Thornes,
Olenna Redwyne, and grandmother of Margaery, has purchased two
exquisite tables from Bravos for the event for the guests to sit and
attend the feast. These tables are big enough to accommodate all $n$
guests but she bought two of them because it is better if some of the
guests are separated. Weddings can lead to quite unfortunate events in
the land of Westeros...\\

\noindent In particular, the list of $n$ attendees includes members of
the Tyrell family of Highgarden, members of the Martell family of
Dorne as well as members of the Lannister family of Casterly
Rock. Unfortunately, not all attendees have good relationships one
with another. For instance, Oberyn Martell accuses Gregor Clegane from
the Lannisters for the death of his sister and her children. So, it
would not be wise to allow both Oberyn and Gregor to sit on the same
table, if possible.\\

\noindent Lord Tywin and the Queen of Thornes have drafted a list of
all $m$ pairs of attendees who do not have good relationships and
should better be placed on different tables. Their objective is to
find an assignment of attendees to the two tables such that the total
number of conflicts between any two guests from different tables is
maximized.\\

\noindent Based on the above description:

\begin{enumerate}
\item \textbf{Formulate the problem that Lord Tywin and the Queen of Thornes
  are facing as a graph problem and define the cost function clearly.}
\item \textbf{Propose a local search algorithm that is a 2-approximation
  polynomial time solution. Show that the algorithm provides a
  2-approximation solution.}
\item \textbf{Propose a greedy algorithm that is a  2-approximation
  polynomial time solution. Show that the algorithm provides a
  2-approximation solution.}
\item \textbf{Propose a randomized algorithm that is a 2-approximation
  polynomial time solution. Show that the algorithm provides a
  2-approximation solution.}
\end{enumerate}

\noindent [Note: Though the problem is similar to the problem 4 in
  homework 3, the assignment here does not require that there is no
  conflict between guests placed on the same table. You may have a
  situation where it is impossible to assign the guests so that no
  conflict arises on a table. For instance, both Oberyn Martell and
  Willas Tyrell hate Gregor Clegane, but they are also not very fond
  of one another... Instead, you are required to find an assignment
  that maximizes the number of conflicts between guests sitting in
  different tables.]\\

\subsection*{Problem 4 (40 points):} Daenerys Targaryen is moving
her army of elite warriors, the Unsullied, in the land of Essos trying
to liberate as many cities as possible from the slave-owners that rule
them. The major cities of Essos, such as Yunkai, Meereen and Astapor,
are trying to coordinate their defenses.\\

\noindent The strategy suggested by Daenerys' advisor Ser Barristan
Selmy is to try to isolate the major cities one from another so that
they cannot communicate and coordinate against Daenerys. Ser Jorah
Mormont generated for this purpose a map of Essos that provides the
road network connecting the biggest $n$ cities on the continent.\\

\noindent Daenerys agreed with Ser Barristan Selmy and decided to
place teams of 100 Unsullied along different road segments so that it
would not be possible for a messenger from one of the major cities to
reach another through the road network without being detected
and... facing the consequences...  Daenerys is worried, however, of
spending too many of her Unsullied for this purpose and reducing her
main military force. So, she wants to identify the minimum number of
road segments that her forces will have to guard so as to achieve this
objective.\\

\noindent Without knowing how the road network with the $n$ cities and
the roads connecting them looks like:

\begin{itemize}
\item[a.] \textbf{Provide a polynomial time algorithm that solves Daenerys'
  problem exactly, when the objective is to separate only Yunkai and
  Astapor.}

\item[b.] \textbf{Provide a solution that achieves an approximation ratio of
  at most 2 when Daenerys wants to separate all three of Yunkai,
  Meereen and Astapor.}
  
\item[c.] \textbf{Propose a local search approach for the general case where
  Daenerys wants to make sure that at least $m$ cities remain isolated
  one from another.}

\item[d.] \textbf{The cities of Essos are working on a counter-strategy after
  they heard the news regarding Daenerys' strategy from spies.} \\

 \noindent They want to place soldiers along the minimum set of roads
 that need to be protected so that all the cities can still
 communicate one with another. When a soldier detects the 100
 Unsullied along a road segment, they will send a large enough force
 to take them out. To be able to monitor a road, it is necessary to
 place a soldier along every mile of the corresponding road. The
 cities need to minimize the number of soldiers used for this purpose
 so that they are sufficiently defended in case Daenerys attacks them
 directly. Consequently, the cost of protecting a longer road is
 directly related to its length [Hint: Thus, the cost of selecting a
   road segment to monitor is a metric function and satisfies the
   triangular inequality.].\\

 \noindent Assume that the road network between the $n$ cities of
 Essos is a complete graph. Consider the problem of identifying the
 minimum set of roads that need to be monitored so that at least the
 $m$ major cities will remain connected given the above strategy.
 Provide a polynomial time approximation algorithm for this problem
 that has an approximation ratio of 2.

\end{itemize}


\end{document}
