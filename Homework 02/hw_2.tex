\documentclass[11pt]{article}

\usepackage{fullpage}
\usepackage[activate={true,nocompatibility},final,tracking=true,kerning=true,spacing=true,factor=1100,stretch=10,shrink=10]{microtype}
\usepackage{multirow}
\usepackage{mathtools}
\usepackage{graphicx}
\usepackage{tabularx}
\usepackage{enumerate}
\usepackage{amsmath, amssymb, graphics, setspace}
\usepackage[noline,noend,ruled,linesnumbered]{algorithm2e}
\usepackage{algpseudocode}
\usepackage{enumitem}

\DeclareMathOperator{\lcm}{lcm}

\newcommand{\mathsym}[1]{{}}
\newcommand{\unicode}[1]{{}}

\title{Assignment 2}

\author{Paul Jones and Matthew Klein \\
		Professor Professor Kostas Bekris\\
		Design and Analysis of Computer Algorithms (01.198.344)}

\date{\today}

\usepackage{setspace}
\singlespacing

\begin{document}

\maketitle

\pagebreak

\section*{Divide-and-Conquer Algorithms, Sorting Algorithms, Greedy Algorithms}


\subsection*{Part A (20 points)}

\subsubsection*{Problem 1}

The more general version of the Master
Theorem is the following. Given a recurrence of the form: 
$$T(n) = a T\left(\frac{n}{b}\right) + f(n)$$
where $a \geq 1$ and $b > 1$ are constants and $f(n)$ is an
asymptotically positive function, there are 3 cases: 
\begin{enumerate}
\item If $f(n) = O(n^{log_ba - \epsilon}$) for some constant $\epsilon
  > 0$, then $T(n) = \Theta(n^{log_ba})$.
\item If $f(n) = \Theta(n^{log_ba} log^kn)$ with $k \geq 0$, then
  $T(n) = \Theta(n^{log_ba} log^{k+1}n)$. In most cases, $k = 0$.
\item If $f(n) = \Omega(n^{log_ba+\epsilon})$ with $\epsilon > 0$, and
  $f(n)$ satisfies the regularity condition, then $T(n) = \Theta( f(n)
  )$. The regularity condition specifies that $a f(\frac{n}{b}) \leq c
  f(n)$ for some constant $c < 1$ and all sufficiently large $n$.
\end{enumerate}

\noindent Give asymptotic bounds for the following recurrences. Assume
$T(n)$ is constant for $n = 1$. Make your bounds as tight as possible,
and justify your answers.\\

\begin{enumerate}[label=\Alph*.]
	\item $T(n) = 2T(\frac{n}{4}) + n^{0.51}$
	
	\begin{itemize}
	
		\item $a = 2$, $b = 4$, where $a \geq 1$ and $b > 1$
		\item $f(n) = \Omega(n^c)$ where $c = 0.51$
		\item $\log_b a = \log_4 2 = \frac{1}{2}$
		\item $c > \log_b a$, that is $0.51 > \frac{1}{2}$
		\item $\frac{n^{0.51}}{2} \leq c n^{0.51}$ for any constant $c < 1$ and $c > 0.5$
		\item $T (n) = \Theta(f(n)) = \Theta (n^{0.51})$
	
	\end{itemize}
	
	\item $T(n) = 16 T(\frac{n}{4}) + n!$
	
	I don't know.
	
	\item $T(n) = \sqrt{2} T(\frac{n}{2}) + \log n$
	
	I don't know.
	
	\item $T(n) = T(n-1) + \log n$
	
	I don't know.
	
	\item $T(n) = 5T(\frac{n}{5}) + \frac{n}{\lg n}$
	
	I don't know.
	
\end{enumerate}

\subsection*{Part B (25 points)}

\subsubsection*{Problem 2} 

You are in the HR department of a
technology firm, and here is a job for you.  There are $n$ different
projects, and $n$ different programmers.

Every project has its unique payoff when completed and level of
difficulty (which are uniform, regardless which programmer will work
on the project).  Every programmer has a unique skill set as well as
expectations for compensation (which are uniform, regardless the
project the programmer will work on). You cannot directly collect
information that allows you to compare the payoff or difficulty level
of two projects, or the capability or expectations for compensation of
two programmers.

Instead, you can arrange a meeting between each project manager and
programmer.  In each meeting, the project manager will give the
programmer an interview to see whether the programmer can do the
project; the programmer can ask the project manager about the
compensation to see whether it meets her expectations.  After the
meeting, you can get a result based on the feedback of the project
manager and the programmer.  The result can be:

\begin{enumerate}
  \item The programmer can't do the project.
  \item The programmer can do the project, but the compensation of the
    project doesn't meet her expectation.
  \item The programmer can do the project, and the compensation for
    the project matches her expectations. At this time, we say the
    project and the programmer \textit{match} with each other.
\end{enumerate}

Assume that the projects and programmers match one to one. Your goal
is to match each programmer to a project.

\begin{enumerate}[label=\Alph*.]
\item \textbf{ Show that any algorithm for this problem must need $\Omega( n
\log{n} )$ meetings in the worst case.}

There are a total of $n$ projects and $n$ programmers. Since each programmer can only work on one job (regardless of outcome), and there are a total of $3$ outcomes. This gives us part of Master's Theorem: $T(n) = a$ where $a = 3$. We need to find the size of each project, which we can guess to be $b = 3$ because we know the answer already ($n \log n$). In fact, this answer makes sense because based on the responses of the programmer's we can find what ones they can do and what ones they can't. We can assume the cost done outside of the calls is instant (meaning talking to the manager's, we aren't concerned with this). Per Master's Theorem: $T(n) = 3(\frac{n}{3}) + 1 \to \log_3^3 = 1$ which is true, so our answer is $n^1 \log n$.

\item  \textbf{Design a randomized algorithm for this problem that runs in
expected time $O ( n \log{n} )$.}

Our randomized algorithm can just be merge sort, we group the programmers in set's, where one set can't do the problem, one set can but wants more pay, and the last set can. Based on their responses, we have our groups to "merge". This algorithm is random because worst, average, and best are $O(n \log n)$.
\end{enumerate}



\subsection*{Part C (40 points)}

\subsubsection*{Problem 3} 

A nation-wide programming contest is held
at $k$ universities in North America. The $i^{th}$ university has
$m_{i}$ participants. The total number of participants is $n$, i.e.,
$n = \sum_{i=1}^{k}m_{i}$. In the contest, participants have to write
programs to solve 6 problems. Each problem contains 10 test cases,
each test case is worth 10 points. Participants aim to maximize their
collected points.  

After the contest, each university sorts the scores of participants
belonging to it and submits the grades to the organizer.  Then the
organizer has to collect the sorted scores of participants and provide
a final sorted list for all participants.

\begin{enumerate}

\item \textbf{For each university, how do they sort the scores of participants
  belonging to it? Please briefly describe a comparative sorting
  algorithm that is appropriate for this purpose and a non-comparative
  sorting algorithm that works in this setup.}

 We can use merge sort or quick sort for our comparison based sorting technique. If we want to sort without it, we can use either count or radix because we know the range of the scores.

\item \textbf{How does the organizer sort the scores of participants given $k$
  files, where each file includes the sorted scores of participants
  from a specific university?  Please describe an algorithm with a
  $O(n\log k)$ running time and justify its time complexity.}

  Because each file is already sorted, we need to use a stable sort. Because it is $O(n\log k)$ we must use something such as merge sort or quick sort. Both of these have running time of $O(n \log k)$ in the average cause because each time we split. In this case, there are $n$ participants and $k$ files of an arbitrary size. Because each university sorts it, then the $k$ files must be these universities. All we do is split it for the sorting, and our answer should be $O(n \log k)$.

\item \textbf{Suppose the organizer wants to figure out the participants of
  ranking $r$ among all participants. Given the $k$ sorted files, how
  does the organizer find the $r^{th}$ largest scores without sorting
  the scores of all participants?  Please describe an algorithm with
  $O(k (\sum_{i=1}^{k}\log m_{i}) )$ running time and justify its time
  complexity. Can you do this in $O(\log k(\sum_{i=1}^{k}\log m_{i}))$ time?}

I don't know.

\end{enumerate}

\subsubsection*{Problem 4} 

You have a collection of $n$ New York Times
crossword puzzles from 01/01/1943 until 12/31/2012 stored in a
database. The only operations that you can perform to the database are
the following:
\begin{itemize}
\item crossword\_puzzle $x$ $\leftarrow$ getPuzzle( int index ); where the
  index is between $1$ and $n$; the puzzles are \emph{not} sorted in the
  database in terms of the date they appeared.
\item getDay( crossword\_puzzle x ); which returns a number between 1
  to 31.
\item getMonth( crossword\_puzzle x ); which returns a number between
  1 to 12.
\item getYear( crossword\_puzzle x ); which returns a number between
  1943 to 2012.
\end{itemize}
All of the above queries can be performed in constant time. You have
found out that the number of puzzles is less than the number of days
in the above period (from 01/01/1943 until 12/31/2012) by one, i.e.,
one crossword puzzle was not included in the database. We need to
identify the date of the missing crossword puzzle.

\textbf{Design a linear-time algorithm that minimizes the amount of space that
it is using to find the missing date. Ignore the effect of leap years.}

\emph{Prima facie}, two good candidates for this task is \emph{counting sort}
and \emph{radix sort}. The reason being, both of them sport linear worst-case
space complexity. However, due to the numerical nature of the problem and the
need for a linear-time worst-case time complexity, lets picks \emph{counting
sort} for our problem.

Once we have our elements sorted, it will be a matter of traversing them to find
the one which does not have its consecutive adjacent to it.

So what does counting sort need? It needs an \emph{input}, our collection of 
New York Time crossword puzzles from 01/01/1943 until 12/31/2012. It needs the
length of the input, that is $n$. It needs a number $k$ such that no number can
exceed this number in the range. And in terms of space complexity it's going
to need a \emph{count array} and an \emph{output array}, along with a few other
small variables.

The way you would execute this algorithmically is you would need a hash function
which would map the first date in the sequence to zero and the last date in the
sequences to the number of dates in the sequence, that is $n$. Then, you would
implement basic counting sort which would accept this $n$ as well as an filled
array of crosswords (that is $n - 1$ in length). This would sort the crossword
in linear time based on it's array with values between $0$ and $n$ with all
a presumed count of $1$ (except our missing one). Then, you would loop over that
sorted list and check to see if any date is not followed by it's consecutive date,
and you've found our missing one.

Here is a specific instantiation of the general guidelines I outlined above:

\begin{verbatim}

hash(crossword) {
    // This function will hash 01/01/1943 to 0, 01/02/1943 to 2, ... 
    // and 12/31/2012 to n. As it turns out, our n is 25567.
    
    // Hashing and normalizing date objects is possible in most or all 
    // programming languages, and I'm not implementing it here to stay
    // language-agnostic.
    
    // This is where I would use getDay, getMonth, and getYear.
    // Here is a crude way of describing the needed code:
    
    if (crossword.getYear() == 1943) {
       if(crossword.getMonth() == 1) {
          if (crossword.getDay() == 1) {
             return 0;
          }
          if (crossword.getDay() == 2) {
             return 1;
          }
          if (crossword.getDay() == 3) {
             return 2;
          }
          ...
       }
       ...
    }
    ...
    
    return x;
}

countingSortForCrosswords(crosswords, n) {
   total = 0;   # variable used in algorithm
   output[n];   # the output, that is sorted, array
   count[n];    # used as a histogram 
   oldCount[n]; # used to transfer values

   # this is the histogram step, and will be presumably 1
   # except for our missing crossword!
   
   for x in crosswords {
      count[hash(x)] += 1;
   }
   
   # this is the starting index step, notice that our maximum
   # in the range is the same as the number of elements
   
   for (int i = 0; i < n; i++) {
      oldCount = count[i];
      count[i] = total;
      total += oldCount;
   }
   
   # copy to output array step
   
   for x in crosswords {
      output[count[hash(x)]] = x;
      count[hash(x)] += 1;
   }
   
   return output;
}

# returns the crossword before the missing one
main(n) {
   crosswords[n];
   for (int i = 0; i < n; i++) {
      crosswords[i] = getPuzzle(i);
   }
   
   countingSortForCrosswords(crosswords, n);
   
   for (int i = 0; i < n - 1; i++) {
      if (hash(crosswords[i]) != hash(crosswords[i + 1])) {
         return i;
      }
   }
}

\end{verbatim}

This will take linear time and require linear space. The reason for it taking
linear time is that all the components of our algorithm are linear or constant.
The hash function is going to be constant, as good hash functions are. That
constant hash function uses only the constant functions that are provided to me.
The counting sort algorithm will also be constant, each of my for loops operates
a total of $n$ times. Finally, my main function will run all it's for loops
in constant time relative to $n$, where the initialization loop will run as many
times as there are crosswords, my sort will be linear, and my check will be linear.
That is,

$$\Omega(n + (n + n) + n) = O(n)$$

Furthermore, it will require linear space because the crosswords are cached
into an array of size $n$, the counting sort utilizes some variables and
cache arrays of size $n$, and that's all I use.

$$O(n)$$

\subsection*{Part D (25 points)}

\subsubsection*{Problem 5} 

You are running a promotional event for a
company during which the plan is to distribute $n$ gifts to the
participants. Consider that each gift $i$ is worth an integer number
of dollars $a_i$. There are $m$ people participating in the event,
where $m < n$. The $j$-th person is satisfied if he receives gifts
that are worth at least $s_j$ dollars each. The task is to satisfy as
many people as possible given that you have a knowledge of the gift
amounts $a_i$ and the satisfaction requirements of each person
$s_j$. 

\begin{itemize}
\item \textbf{Give an approximation algorithm for assigning rewards to people
with a running time of $O(m\log m+n)$.} 

I don't know.

\item \textbf{What is the approximation ratio of your algorithm and why?}

I don't know.

\end{itemize}

\end{document}

