\documentclass[11pt]{article}

\usepackage{fullpage}
\usepackage[activate={true,nocompatibility},final,tracking=true,kerning=true,spacing=true,factor=1100,stretch=10,shrink=10]{microtype}
\usepackage{multirow}
\usepackage{mathtools}
\usepackage{graphicx}
\usepackage{tabularx}
\usepackage{enumerate}
\usepackage{amsmath, amssymb, graphics, setspace}
\usepackage[noline,noend,ruled,linesnumbered]{algorithm2e}
\usepackage{algpseudocode}
\usepackage{enumitem}

\DeclareMathOperator{\lcm}{lcm}

\newcommand{\mathsym}[1]{{}}
\newcommand{\unicode}[1]{{}}

\title{Assignment 2}

\author{Paul Jones and Matthew Klein \\
		Professor Professor Kostas Bekris\\
		Design and Analysis of Computer Algorithms (01.198.344)}

\date{\today}

\usepackage{setspace}
\singlespacing

\begin{document}

\maketitle

\pagebreak

\section*{Divide-and-Conquer Algorithms, Sorting Algorithms, Greedy Algorithms}


\subsection*{Part A (20 points)}

\subsubsection*{Problem 1}

The more general version of the Master
Theorem is the following. Given a recurrence of the form: 
$$T(n) = a T\left(\frac{n}{b}\right) + f(n)$$
where $a \geq 1$ and $b > 1$ are constants and $f(n)$ is an
asymptotically positive function, there are 3 cases: 
\begin{enumerate}
\item If $f(n) = O(n^{log_ba - \epsilon}$) for some constant $\epsilon
  > 0$, then $T(n) = \Theta(n^{log_ba})$.
\item If $f(n) = \Theta(n^{log_ba} log^kn)$ with $k \geq 0$, then
  $T(n) = \Theta(n^{log_ba} log^{k+1}n)$. In most cases, $k = 0$.
\item If $f(n) = \Omega(n^{log_ba+\epsilon})$ with $\epsilon > 0$, and
  $f(n)$ satisfies the regularity condition, then $T(n) = \Theta( f(n)
  )$. The regularity condition specifies that $a f(\frac{n}{b}) \leq c
  f(n)$ for some constant $c < 1$ and all sufficiently large $n$.
\end{enumerate}

\noindent Give asymptotic bounds for the following recurrences. Assume
$T(n)$ is constant for $n = 1$. Make your bounds as tight as possible,
and justify your answers.\\

\begin{enumerate}[label=\Alph*.]
	\item $T(n) = 2T(\frac{n}{4}) + n^{0.51}$
	
	\begin{itemize}
	
		\item $a = 2$, $b = 4$
		\item $f(n) = \Omega(n^c)$ where $c = 0.51$
		\item $\log_b a = \log_4 2 = \frac{1}{2}$
		\item $c > \log_b a$, that is $0.51 > \frac{1}{2}$
		\item $\frac{n^{0.51}}{2} \leq c n^{0.51}$ for any constant $c < 1$ and $c > 0.5$
		\item $T (n) = \Theta(f(n)) = \Theta (n^{0.51})$
	
	\end{itemize}
	
	\item $T(n) = 16 T(\frac{n}{4}) + n!$
	
	\begin{itemize}
	
		\item $a = 16$, $b = 4$
		\item $f(n) = n!$
	
	\end{itemize}
	
	\item $T(n) = \sqrt{2} T(\frac{n}{2}) + \log n$
	
	\begin{itemize}
	
		\item $a = \sqrt{2}$, $b = 2$
		\item $f(n) = \log n$
	
	\end{itemize}
	
	\item $T(n) = T(n-1) + \log n$
	
	\begin{itemize}
	
		\item $a = 1$, $b = 1$
		\item $f(n) = \log n$
	
	\end{itemize}
	
	\item $T(n) = 5T(\frac{n}{5}) + \frac{n}{\lg n}$
	
	\begin{itemize}
	
		\item $a = 5$, $b = 5$
		\item $f(n) = \log n$
	
	\end{itemize}
	
\end{enumerate}

\subsection*{Part B (25 points)}

\subsubsection*{Problem 2} 

You are in the HR department of a
technology firm, and here is a job for you.  There are $n$ different
projects, and $n$ different programmers.

Every project has its unique payoff when completed and level of
difficulty (which are uniform, regardless which programmer will work
on the project).  Every programmer has a unique skill set as well as
expectations for compensation (which are uniform, regardless the
project the programmer will work on). You cannot directly collect
information that allows you to compare the payoff or difficulty level
of two projects, or the capability or expectations for compensation of
two programmers.

Instead, you can arrange a meeting between each project manager and
programmer.  In each meeting, the project manager will give the
programmer an interview to see whether the programmer can do the
project; the programmer can ask the project manager about the
compensation to see whether it meets her expectations.  After the
meeting, you can get a result based on the feedback of the project
manager and the programmer.  The result can be:

\begin{enumerate}
  \item The programmer can't do the project.
  \item The programmer can do the project, but the compensation of the
    project doesn't meet her expectation.
  \item The programmer can do the project, and the compensation for
    the project matches her expectations. At this time, we say the
    project and the programmer \textit{match} with each other.
\end{enumerate}

Assume that the projects and programmers match one to one. Your goal
is to match each programmer to a project.

{\bf A.} Show that any algorithm for this problem must need $\Omega( n
\log{n} )$ meetings in the worst case.

{\bf B.}  Design a randomized algorithm for this problem that runs in
expected time $O ( n \log{n} )$.

\subsection*{Part C (40 points)}

\subsubsection*{Problem 3} 

A nation-wide programming contest is held
at $k$ universities in North America. The $i^{th}$ university has
$m_{i}$ participants. The total number of participants is $n$, i.e.,
$n = \sum_{i=1}^{k}m_{i}$. In the contest, participants have to write
programs to solve 6 problems. Each problem contains 10 test cases,
each test case is worth 10 points. Participants aim to maximize their
collected points.  

After the contest, each university sorts the scores of participants
belonging to it and submits the grades to the organizer.  Then the
organizer has to collect the sorted scores of participants and provide
a final sorted list for all participants.

\begin{enumerate}

\item For each university, how do they sort the scores of participants
  belonging to it? Please briefly describe a comparative sorting
  algorithm that is appropriate for this purpose and a non-comparative
  sorting algorithm that works in this setup.

\item How does the organizer sort the scores of participants given $k$
  files, where each file includes the sorted scores of participants
  from a specific university?  Please describe an algorithm with a
  $O(n\log k)$ running time and justify its time complexity.

\item Suppose the organizer wants to figure out the participants of
  ranking $r$ among all participants. Given the $k$ sorted files, how
  does the organizer find the $r^{th}$ largest scores without sorting
  the scores of all participants?  Please describe an algorithm with
  $O(k (\sum_{i=1}^{k}\log m_{i}) )$ running time and justify its time
  complexity. 

Can you do this in $O(\log k(\sum_{i=1}^{k}\log m_{i}))$ time?

[Hint: If $r$ is larger than $\dfrac{n}{2}$, the elements that have at
  least $\dfrac{n}{2}$ elements larger than them should not be
  considered. The problem is how to identify these elements within
  each sorted file.]\\
  

\end{enumerate}


\subsubsection*{Problem 4} 

You have a collection of $n$ New York Times
crossword puzzles from 01/01/1943 until 12/31/2012 stored in a
database. The only operations that you can perform to the database are
the following:
\begin{itemize}
\item crossword\_puzzle $x$ $\leftarrow$ getPuzzle( int index ); where the
  index is between $1$ and $n$; the puzzles are \emph{not} sorted in the
  database in terms of the date they appeared.
\item getDay( crossword\_puzzle x ); which returns a number between 1
  to 31.
\item getMonth( crossword\_puzzle x ); which returns a number between
  1 to 12.
\item getYear( crossword\_puzzle x ); which returns a number between
  1943 to 2012.
\end{itemize}
All of the above queries can be performed in constant time. You have
found out that the number of puzzles is less than the number of days
in the above period (from 01/01/1943 until 12/31/2012) by one, i.e.,
one crossword puzzle was not included in the database. We need to
identify the date of the missing crossword puzzle.

Design a linear-time algorithm that minimizes the amount of space that
it is using to find the missing date. Ignore the effect of leap years.

\subsection*{Part D (25 points)}

\subsubsection*{Problem 5} 

You are running a promotional event for a
company during which the plan is to distribute $n$ gifts to the
participants. Consider that each gift $i$ is worth an integer number
of dollars $a_i$. There are $m$ people participating in the event,
where $m < n$. The $j$-th person is satisfied if he receives gifts
that are worth at least $s_j$ dollars each. The task is to satisfy as
many people as possible given that you have a knowledge of the gift
amounts $a_i$ and the satisfaction requirements of each person
$s_j$. Give an approximation algorithm for assigning rewards to people
with a running time of $O(m\log m+n)$. What is the approximation ratio
of your algorithm and why?


\end{document}

