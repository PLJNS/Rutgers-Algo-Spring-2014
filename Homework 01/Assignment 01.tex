\documentclass[11pt]{article}

\usepackage{fullpage}
\usepackage[activate={true,nocompatibility},final,tracking=true,kerning=true,spacing=true,factor=1100,stretch=10,shrink=10]{microtype}
\usepackage{multirow}
\usepackage{mathtools}
\usepackage{graphicx}
\usepackage{tabularx}
\usepackage{enumerate}
\usepackage{amsmath, amssymb, graphics, setspace}

\newcommand{\mathsym}[1]{{}}
\newcommand{\unicode}[1]{{}}

\title{Assignment 01}

\author{Paul Jones and Matthew Klein \\
		Professor Professor Kostas Bekris\\
		Design and Analysis of Computer Algorithms (01.198.344)}

\date{\today}

\usepackage{setspace}
\doublespacing

\begin{document}

\maketitle

\pagebreak

\section*{Part A}

\subsection*{Problem 1}

In each of the following situations indicate whether $f = O(g)$ or $f = \Omega(g)$ 
or $f = \Theta(g)$:

\begin{enumerate}

\item $f(n) = \sqrt{2^{7x}}, g(n) = \lg(7^{2x})$

\[ f(n) = \sqrt{2^{7x}} = \sqrt{128^{x}}\]
\[ g(n) = \lg(7^{2x}) = \lg(49^{x})\]
\[ lg(49^1) \approx 5.6 \]
\[ \sqrt{128^{1}} \approx 11.3 \]

Notice that both of these functions only grow relative to $x$.

\[ f = \Omega(g) \]

\item $f(n) = 2^{nln(n)}, g(n) = n!$

The factorial, that is $n!$, function grows much, much faster than $2^n$.

\[ f = \Omega(g) \]

\item $f(n) = \lg(\lg^*(n)), g(n) = \lg^*(\lg(n)) $

\[ f = \Theta(g) \]

\item $f(n) = \frac{lg(n^2)}{n}, g(n) = lg^*(n)$

\[f(n) = \frac{lg(n^2)}{n} = \frac{2 lg(n)}{n}\]

\[ f = \Theta(g) \]

\item $f(n) = 2^n, g(n) = n^{\lg(n)}$

This is comparing the exponential function to a function that is less than
$n^2$.

\[ f = \Omega(g) \]

\item $f(n) = 2^{\sqrt{\ln(n)}}, g(n) = n(\lg(n)^3)$

\[ f(n) = 2^{\sqrt{n}}, g(n) = (2^n) (n^3) \]

\[f = \Omega{g}\]

\item $f(n) = e^{\cos(x)}, g(n) = \lg(x)$

\[ f = \Omega(g) \]

\item $ f(n) = \lg(n^2), g(n) = (\lg(n))^2 $

\[ f = \Theta(g) \]

\item $ f(n) = \sqrt{4n^2 - 12n + 9}, g(n) = n^{\frac{3}{2}} $

\[ f = \Theta(g) \]

\item $ f(n) = \sum_{k = 1}^{n} k, g(n) = (n + 2)^2 $

\[ f = \Omega(g) \]

\end{enumerate}

\subsection*{Problem 2}

\section*{Part B}

\subsection*{Problem 3}

\subsection*{Problem 4}

\subsection*{Problem 5}

\section*{Part C}

\subsection*{Problem 6}

\subsection*{Problem 7}

\section*{Part D}

\subsection*{Problem 8}

\subsection*{Problem 9}

\end{document}