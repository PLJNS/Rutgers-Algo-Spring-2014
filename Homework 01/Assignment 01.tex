\documentclass[11pt]{article}

\usepackage{fullpage}
\usepackage[activate={true,nocompatibility},final,tracking=true,kerning=true,spacing=true,factor=1100,stretch=10,shrink=10]{microtype}
\usepackage{multirow}
\usepackage{mathtools}
\usepackage{graphicx}
\usepackage{tabularx}
\usepackage{enumerate}
\usepackage{amsmath, amssymb, graphics, setspace}
\usepackage[noline,noend,ruled,linesnumbered]{algorithm2e}
\usepackage{algpseudocode}

\DeclareMathOperator{\lcm}{lcm}

\newcommand{\mathsym}[1]{{}}
\newcommand{\unicode}[1]{{}}

\title{Assignment 01}

\author{Paul Jones and Matthew Klein \\
		Professor Professor Kostas Bekris\\
		Design and Analysis of Computer Algorithms (01.198.344)}

\date{\today}

\usepackage{setspace}
\singlespacing

\begin{document}

\maketitle

\pagebreak

\section*{Part A}

\subsection*{Problem 1}

\textbf{In each of the following situations indicate whether $f = O(g)$ or $f = \Omega(g)$ 
or $f = \Theta(g)$:}

\begin{enumerate}

	\item $f(n) = \sqrt{2^{7x}}, g(n) = \lg(7^{2x})$
		
		\[ f(n) = \sqrt{2^{7x}} = \sqrt{128^{x}}\]
		\[ g(n) = \lg(7^{2x}) = \lg(49^{x})\]
		\[ lg(49^1) \approx 5.6 \]
		\[ \sqrt{128^{1}} \approx 11.3 \]
		
		Notice that both of these functions only grow relative to $x$.
		
		\[ f = \Omega(g) \]
	
	\item $f(n) = 2^{nln(n)}, g(n) = n!$
	
		The factorial, that is $n!$, function grows much, much faster than $2^n$.
		
		\[ f = \Omega(g) \]
	
	\item $f(n) = \lg(\lg^*(n)), g(n) = \lg^*(\lg(n)) $
	
		\[ f = \Theta(g) \]
	
	\item $f(n) = \frac{lg(n^2)}{n}, g(n) = lg^*(n)$
	
		\[f(n) = \frac{lg(n^2)}{n} = \frac{2 lg(n)}{n}\]
		
		\[ f = \Theta(g) \]
		
	\item $f(n) = 2^n, g(n) = n^{\lg(n)}$
	
		This is comparing the exponential function to a function that is less than
		$n^2$.
		
		\[ f = \Omega(g) \]
		
	\item $f(n) = 2^{\sqrt{\ln(n)}}, g(n) = n(\lg(n)^3)$
	
		\[ f(n) = 2^{\sqrt{n}}, g(n) = (2^n) (n^3) \]
		
		\[f = \Omega{g}\]
	
	\item $f(n) = e^{\cos(x)}, g(n) = \lg(x)$
	
		\[ f = \Omega(g) \]
	
	\item $ f(n) = \lg(n^2), g(n) = (\lg(n))^2 $
	
		\[ f = \Theta(g) \]
	
	\item $ f(n) = \sqrt{4n^2 - 12n + 9}, g(n) = n^{\frac{3}{2}} $
	
		\[ f = \Theta(g) \]
	
	\item $ f(n) = \sum_{k = 1}^{n} k, g(n) = (n + 2)^2 $
	
		\[ f = \Omega(g) \]

\end{enumerate}

\subsection*{Problem 2}

\begin{algorithm}[h]
\caption{${\tt Number\_Theoretic\_Algorithm}$ $($ integer $n$ ) }
\label{algo}
$N \leftarrow Random\_Sample( 0, 2^n-1)$\; 
\If{ $N$ is even }
{
  $N \leftarrow N + 1$ \CommentSty{/* Worse case, N is odd, 2 ** N - 1. */} \; 
}
$m \leftarrow N \mod\ n$ \CommentSty{/* worse case same as n */} \; 
\For{ $j \leftarrow 0$ to $m$ }
{
  \If{ ${\tt Greatest\_Common\_Divisor}$( $j, N ) \neq 1$ }
  {
    return {\tt FALSE}; \CommentSty{/* GCD is O(n) */}
  }
  Compute $x,z$ so that $N - 1 = 2^z \cdot x$ and $x$ is odd\;
  $y_0 \leftarrow (N-1-j)^x \mod N$\;
  \For{ $i \leftarrow 1$ to $m$}
  {
    $y_i \leftarrow y_{i-1}^2 \mod N$ \;
    $y_i \leftarrow y_i + y_{i-1} \mod N$\;
  }
  \If{ {\tt Low\_Error\_Primality\_Test}($y_m$) == {\tt FALSE} }
  {
    return {\tt FALSE} \CommentSty{/* Naive primality test is O(sqrt(n)) */} \;
  }
}
return {\tt TRUE}\;
\end{algorithm}

Compute the asymptotic running time of the
above algorithm as a function of its input parameter, given:

\begin{itemize}
\item The running times of integer arithmetic operations (e.g.,
  multiplication of two large $n$-bit numbers is $O(n^2)$).
\item Assume that sampling a number $N$ is an operation linear to the
  number of bits needed to represent this number.
\end{itemize}

\noindent Do not just present the final result. For each line of
pseudo-code indicate the best running time for the corresponding
operation given current knowledge from lectures and recitations and
then show how the overall running time emerges.\\

%First line takes 1 to save and 1 to generate.
%The first if block check is 1 (mod by 2, check if 0) along with 1 for assignment %and (cost of add here) for addition.
%Line 4 takes: , (mod time) for mod and 1 for assignment
%Line 5, the loop lasts m times and for each sub line: the if check is GCD runtime, %the computation is (unknown), y0 is: (cost of mod + exponent)
%the inner loop is m-1 times, and then for both line 11/12 is (modular exponential %time). the last if check is hfthe time for primality test.

Worse case running $n$ operations with times $O(n)$, $O(n)$, and $O(\sqrt{n})$.

That's a run time of $O(2 n^2 + n^{\frac{3}{2}})$, resulting in big-O of $O(n^2)$.

\section*{Part B}

\subsection*{Problem 3}
\begin{itemize}


\item \textbf{Consider that we have a tree data structure $T_m^N$, where every
  node can have at most $m$ children and the tree has at most $N$
  nodes total. Compute a lower bound for the height of the tree.}

A tree with $m$ children is $\log_{m}(N+1)-1$.

\item \textbf{Consider two such trees $T_m^N$ and $T_{m'}^N$ that are
  ``perfect'', i.e., every node has exactly $m$ and $m'$ children
  correspondingly. Now, consider the functions $h_m(N)$ and
  $h_{m'}(N)$ that express the heights of these perfect trees for
  different values of $N$. What is the asymptotic behavior of $h_m$
  relative to $h_{m'}$ and under what conditions?}

A perfect tree will only be changing based on the $m,m^\prime$ values. Whichever value is larger will run faster.

\item \textbf{Consider the following rule for modular exponentiation, where
  $x$ is in the order of $2^m$ and $y$ is in the order of $2^n$. What
  is the running time of computing the result according to this rule?
$$x^y = \left\{
\begin{array}{c l}     
    (x ^{\lfloor \frac{y}{2} \rfloor} )^2,  & \textrm{ if y is even }\\
    x \cdot (x^{\lfloor \frac{y}{2} \rfloor} )^2, & \textrm{ if y is odd}
\end{array}\right.$$
}

\begin{itemize}

\item On the top level, just like multiplication, this algorithm will have at most $n$ recursive calls.
\item During each call it multiplies n-bit numbers, which is in the order of $O(n^2)$.
\item The resulting $O(n \cdot n^2)$ is $O(n^3)$.

\end{itemize}

\end{itemize}


\subsection*{Problem 4}
\begin{itemize}
\item \textbf{Compute the following: $2^{902}$ mod 7.}

I found out how to do this using a website, since I didn't understand how to from lecture ? $2^{902} \bmod 7$ We can find the original, $2 \bmod 7 = 2$ because $7$ doesn't go into $2$ at all. We can next square, finding $4 \bmod 7 = 4$. Divide exponent in half, $2^{451} \bmod 7$. Next we can do $4 \bmod 7 = 4$ again, and square. $16 \bmod 7 = 2$. Once again we cut our exponent, $2^{225} \bmod 7$. Now we have $4 \cdot 2 \bmod 7 \to 8 \bmod 7 = 1$. Next we square our other value, $4 \bmod 7 = 4$. We divide exponent again, $2^{112} \bmod 7$, and we do $16 \bmod 7 = 2$. Another cut, $2^{56} \bmod 7$. We can check $2^2 \bmod 7 = 4$. Another time we cut, $2^{28} \bmod 7$. We need to use previous value again, $16 \bmod 7 = 2$. $2^{14} \bmod 7$ from another cut, and we use $4 \bmod 7 = 4$. We can cut again, $2^7 \bmod 7$ and we use $4 \bmod 7 = 4$. We are almost done and use $2^3 \bmod 7$. We must check $8 \bmod 7 = 1$, and now we are on the final step. $2^1 \bmod 7 = 4$

\item \textbf{Find the modulo multiplicative inverse of 11 mod 120, 13 mod 45,
  35 mod 77, 9 mod 11, 11 mod 1111.}

$11 \bmod 120 = 121, 13 \bmod 45 = 91, 9 \bmod 11 = 45$. For the last one and third one I used Extended Euclidean Algorithm discussed in class. I also used $p_i = p_{i-2} - p_{i-1} q_{i-2} \bmod n$.


Third one: $35 \bmod 77 \to 77 = 2(35) + 7$ and $p_0 = 0$. Next, $35 = 5(7) + 0$ and $p_1 = 1$. However, this can't be solved.

Last one: $11 \bmod 1111 \to 1111 = 101(11) + 0$. This one can't be solved either because we were unable to get past the step, like the third one.  

\item \textbf{Assume that for a number $x$ the following property is true:
  $\forall\ y \in [1,x-1]:\ gcd(x,y) = 1$. Compute the running time of
  an efficient algorithm for finding all the inverses modulo $x^m$
  from the set $\{0, 1, \ldots, x^m-1\}$ that exist.}
  
$\forall y \in [1,x-1]: \gcd(x,y) = 1$. If we want to find all of the modulo $x^m$ between $0,1,...,x^m -1$ then we can assume there are $m$ total modulo inverses to compute. An example is that there $x = 2, m = 2$ to keep it simple. This means that every number from $1 \to 1: \gcd(1,1) = 1$ which is correct. Now we need to find $0,...,2^2-1$ which becomes $0,...,3$. We have a total of $4$ numbers to modulo inverse. The running time to find is the amount multiplied by the time it takes to run the euclidean algorithm. There's a total of $x^m$ to find and the Extended Euclidean algorithm takes $\log(m^2)$. Our total runtime is $x^m \log(m^2)$.

\end{itemize}

\subsection*{Problem 5}

\begin{itemize}
\item \textbf{Assume two positive integers $x < y$. Then the pairs
  $(5x+3y,3x+2y)$ and $(x,y)$ have the same greater common
  divisor. True or False, explain.}

\begin{itemize}

\item The way we can solve this is using the following lemma, where for
all integers and numbers greater than zero:

$$ \gcd(x, y) = \frac{xy}{\lcm(x, y)} $$

\item Notice that the left half of the equation will serve as an answer
to the the second pair in the given problem. 

\item Witness that the lowest common multiple of $x$ and $y$ is $xy$.

\item For the first pair, witness that by the same rule:

$$ \gcd((5x+3y), (3x+2y)) = \frac{(5x+3y)(3x+2y)}{\lcm((5x+3y), (3x+2y))} $$

\item Witness that the lowest common multiple of $(5x+3y)$ and $(3x+2y)$ is $(5x+3y)(3x+2y)$.

\item Multiply

$$(5x+3y)(3x+2y) = 15x^2 + 19xy + 6y^2$$

\item Notice that the claim being tested by the problem can be symbolized
by the following:

$$ \forall x \forall y (x < y \land x,y \ne 0 \land x,y \in \mathbb{Z} \land  \gcd((5x+3y), (3x+2y)) = \gcd(x, y)) $$

\item Just consider the last condition:

$$ \gcd((5x+3y), (3x+2y)) = \gcd(x, y) $$

\item Substitute the equalities we found earlier:

$$ \frac{(5x+3y)(3x+2y)}{\lcm((5x+3y), (3x+2y))} = \frac{xy}{\lcm(x, y)} $$

\item Substitute the lowest common multiples we found earlier:

$$ \frac{(5x+3y)(3x+2y)}{15x^2 + 19xy + 6y^2} = \frac{xy}{xy} $$

\item Multiply

$$ \frac{15x^2 + 19xy + 6y^2}{15x^2 + 19xy + 6y^2} = \frac{xy}{xy} $$

\item Simplify

$$ 1 = 1 $$

\item It's true.

\end{itemize}

\item \textbf{Consider the following sequence of numbers: $s_n = 1 +
  \Pi_{i=0}^{n-1} s_i$, where $s_0 = 2$. Prove that any two numbers in
  this sequence are relatively prime.}
\end{itemize}

\section*{Part C}

\subsection*{Problem 6}

\begin{itemize}

\item \textbf{A proof that the hash function family $\mathcal{M}$ is
  universal.}


The hash function for the family is definitely consistent because each item is only either $0,1$, and we are modding by the total amount of choices, but I'm not sure how to prove this other than by what was stated in class: $Pr = { h_\alpha(x) = h_\alpha(y)} = Pr { \sum_{i=1}^4 \alpha_i \cdot x_i = \sum_{i = 1}^4 \alpha_i \cdot y_i \bmod N} = Pr {\underbrace{\sum_{i=1}^3 \alpha_i (x_i - y_i) = \alpha_4 (y_4 - x_4 \bmod N)}_\text{given x, y and randomly picked a1,a2,a3: c is constant}}$. However, in this case this would be different because we have $0,1$ not $1,...,4$: $Pr = { h_\alpha(x) = h_\alpha(y)} = Pr { \sum_{i=1}^2 \alpha_i \cdot x_i = \sum_{i = 1}^2 \alpha_i \cdot y_i \bmod N} = Pr {\underbrace{\sum_{i=1}^1 \alpha_i (x_i - y_i) = \alpha_2 (y_2 - x_2 \bmod N)}_\text{given x, y and randomly picked a1: c is constant}}$.

\item \textbf{A comparison to the universal hash function family described in
  DPV chapter 1.5.2. How many random bits are needed here?}

\end{itemize}

\subsection*{Problem 7}

\begin{itemize}

\item \textbf{Assume that number $n$ is prime, then all numbers $1 \leq x < n$
  are invertible modulo $n$. Which of these numbers are their own
  inverse module $n$?}

\item \textbf{Show that $(n-1)! \equiv -1 \mod n$) for prime $n$. [Hint:
  Compute the value of $(n-1) \mod n$ by considering how it arises by
  pairing two numbers smaller than $n$ that are multiplicative
  inverses modulo $n$.]}

\item \textbf{Show that if $n$ is not prime, then $(n-1)! \neq -1 \mod
  n$. [Hint: What does it mean that $n$ is not prime in terms of the
    numbers $1 \leq x < n$?]}

\item \textbf{The above process can be used as a primality test instead of
  Fermat's Little theorem as it is an if-and-only-if condition for
  primality. Why can't we immediately base a primality test on this
  rule? [Tip: Even if you are not able to answer the previous two
    questions, you should be able to argue about this question.]}

\end{itemize}

\section*{Part D}

\subsection*{Problem 8}

\subsubsection*{Part A} 

\textbf{Make a table with three columns. The first column is all numbers
from 0 to 36. The second is the residues of these numbers modulo 5;
the third column is the residues modulo 7.}

\subsubsection*{Part B} 

\textbf{Consider two different prime numbers $x$ and $y$. Show that the
following is true: for every pair of numbers $m$ and $n$ so that: $0
\leq m < x$ and $0 \leq n < y$, there is a unique integer $q$, where
$0 \leq q < xy$, so that:}

$$q \equiv m \pmod{x}$$
$$q \equiv n \pmod{y}$$
  
\subsubsection*{Part C}

\textbf{The previous problem asks to go from $q$ to $(m,n)$. It is also
possible to go the other way. In particular, show the following:}

$$q = ( m \cdot y \cdot (y^{-1} \bmod{x}) + n \cdot x \cdot
(x^{-1} \bmod{y}) ) \bmod{xy}$$



\subsubsection*{Part D} 

\textbf{What happens in the case of three primes $x$, $y$ and $z$? Do the
above properties still hold? If they do, how do they look like in this
case?}

\subsection*{Problem 9}

\begin{itemize}

\item \textbf{How did Mallory do this? }

\item \textbf{What is the original message?}

\end{itemize}

\end{document}